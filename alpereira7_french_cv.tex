%%%%%%%%%%%%%%%%%%%%%%%%%%%%%%%%%%%%%%%%%
% Medium Length Professional CV
% LaTeX Template
% Version 2.0 (8/5/13)
%
% This template has been downloaded from:
% http://www.LaTeXTemplates.com
%
% Original author:
% Trey Hunner (http://www.treyhunner.com/)
%
% Important note:
% This template requires the resume.cls file to be in the same directory as the
% .tex file. The resume.cls file provides the resume style used for structuring the
% document.
%
%%%%%%%%%%%%%%%%%%%%%%%%%%%%%%%%%%%%%%%%%

%----------------------------------------------------------------------------------------
%	PACKAGES AND OTHER DOCUMENT CONFIGURATIONS
%----------------------------------------------------------------------------------------

\documentclass{resume} % Use the custom resume.cls style

\usepackage[left=0.6in,top=0.5in,right=0.6in,bottom=0.4in]{geometry} % Document margins

\name{alpereira7} % Your name

\begin{document}

%----------------------------------------------------------------------------------------
%	WORK EXPERIENCE SECTION
%----------------------------------------------------------------------------------------
	
\begin{rSection}{Expérience}

\begin{rSubsection}{Elno (Communications Défense et Sécurité)}{Depuis Septembre 2018}{Ingénieur Développement et Traitement du Signal}{Argenteuil}
\item Développement d'algorithmes audio
\item Portage d'algorithme sur DSP
\item Optimisation du code sous contraintes, tests unitaires, validations
\item Rédaction des documentations
\end{rSubsection}

%------------------------------------------------

\begin{rSubsection}{Acoustical Beauty}{Juillet 2018 - Août 2018}{Développeur Freelance d'Algorithmes}{Saint-Pierre-du-Perray}
\item Développement d'algorithmes de sur-échantillonnage audio
\end{rSubsection}

%------------------------------------------------

\begin{rSubsection}{Arkamys (Audio embarqué, automobile)}{Avril 2016 - Juin 2018}{Ingénieur Traitement du Signal}{Paris}
\item Développement et mise à jour d'algorithmes de traitement audio
\item Portage d'algorithmes sur un DSP
\item Optimisation du code sous contraintes, tests unitaires, validations
\end{rSubsection}

%------------------------------------------------

\begin{rSubsection}{DMS/Auro Technologies (Cinéma, home-cinéma)}{Juillet 2012 - Février 2016}{Ingénieur R\&D et Traitement du Signal}{Lognes}
\item Implémentation d'un code binaural
\item Représentant de DMS auprès du consortium BiLi (Binaural Listening)
\item Développement d'algorithmes de post-processing
\item Suivi des certifications Dolby et DTS
\item Rédaction des documentations, tests unitaires et validations
\end{rSubsection}

%------------------------------------------------

\begin{rSubsection}{CADLM}{Janvier 2011 - Juillet 2012}{Ingénieur Recherche et Développement}{Massy}
\item Calcul Scientifique, Analyse Numérique et Intelligence Artificielle
\item Développement d'outils mathématiques
\item Implémentation de réseaux de neurones et conception de librairies mathématiques en Python
\item Développement d'outils de datamining
\end{rSubsection}

%------------------------------------------------

\begin{rSubsection}{LIMSI-CNRS (Mécanique des fluides, sciences de l'ingénieur, aéro)}{Septembre 2009 - Août 2010}{Ingénieur d'études}{Orsay}
\item Projet 1 : prise en compte des incertitudes
\item Développement des librairies stochastiques
\item Implémentation des incertitudes dans le code principal \smallskip
\item Projet 2 : Assemblage de l'opérateur Laplacien
\item Conception et développement d'une méthode de maillage de type "arbre"
\end{rSubsection}

%------------------------------------------------

\begin{rSubsection}{BRGM (Industrie, sous-sols, risques naturels)}{Avril 2009 - Août 2009}{Ingénieur Stagiaire}{Orléans}
\item Modélisation hydromécanique du stockage géologique du CO2
\item Couplage de codes, tests et validations
\end{rSubsection}

%------------------------------------------------

\begin{rSubsection}{Université Paris-Sud}{2008}{Projet}{Orsay}
\item Résolution des équations de Laplace et de la chaleur par la méthode des éléments finis
\end{rSubsection}

%------------------------------------------------

\begin{rSubsection}{ONERA (Aéronautique, Défense)}{Avril 2008 - Juillet 2008}{Ingénieur Stagiaire}{Palaiseau}
\item Modélisation de la réfraction atmosphérique par un schéma numérique à pas de discrétisation variable
\end{rSubsection}

\end{rSection}

%----------------------------------------------------------------------------------------
%	EDUCATION SECTION
%----------------------------------------------------------------------------------------

\begin{rSection}{Formations}

{\bf Conservatoire National des Arts et Métiers, Paris} \hfill {\em 2018} \\
Bases de Transmissions Numériques (I) (ELE-112) \\
Bases de Traitement du Signal (ELE-103) \\
Traitement Numérique du Signal (ELE-102)

{\bf Université Paris-Sud 11, Orsay} \hfill {\em 2009} \\ 
Master en Ingénierie Mathématique \smallskip \\
Calcul scientifique, Modélisation/simulation numérique, optimisation

\end{rSection}

%----------------------------------------------------------------------------------------
%	CERTIFICATIONS
%----------------------------------------------------------------------------------------

\begin{rSection}{Certifications MOOC}

{\bf IMT, FUN-MOOC} \hfill {\em 2020} \\
Arduino

{\bf Stanford, Coursera} \hfill {\em 2019} \\ 
Machine Learning

{\bf Open Classrooms} \hfill {\em 2016} \\ 
Linux, C, C++

\end{rSection}

%----------------------------------------------------------------------------------------
%	TECHNICAL STRENGTHS SECTION
%----------------------------------------------------------------------------------------

\begin{rSection}{Compétences Techniques}

\begin{tabular}{ @{} >{\bfseries}l @{\hspace{6ex}} l }
Langages & C, Python (NumPy, SciPy), Matlab, Octave \\
Tools & SVN, Git, LibSndFile, \LaTeX, \\
Processeur & SHARC (ADSP-21469, ADSP-21489) \\
Microcontroller & Microchip PIC, Arduino
\end{tabular}

\end{rSection}

%----------------------------------------------------------------------------------------
%	LANGUES
%----------------------------------------------------------------------------------------

\begin{rSection}{Langues}

\begin{tabular}{ @{} >{\bfseries}l @{\hspace{6ex}} l }
Anglais & Courant \\
Portugais & Langue Maternelle
\end{tabular}

\end{rSection}

%----------------------------------------------------------------------------------------
%	LOISIRS
%----------------------------------------------------------------------------------------

\begin{rSection}{Loisirs}

\begin{tabular}{ @{} >{\bfseries}l @{\hspace{6ex}} l }
GitHub & alpereira7 \\
Musique & Pratique occasionnelle de la basse \\
Lecture & Science-Fiction, Fantasy \\
Dev & Arduino, JUCE, Alsa
\end{tabular}

\end{rSection}

%----------------------------------------------------------------------------------------
%	EXAMPLE SECTION
%----------------------------------------------------------------------------------------

%\begin{rSection}{Section Name}

%Section content\ldots

%\end{rSection}

%----------------------------------------------------------------------------------------

\end{document}
